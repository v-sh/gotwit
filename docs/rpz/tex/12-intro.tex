\Introduction

В настоящее время существует большое разнообразие социальных сетей различных направлений. Основными потребностями пользователей данных продуктов являются:
\begin{itemize}
\item Общение по средством сообщений, комментариев...
\item Получение информации из определенных источников (публикаций других пользователей)
\end{itemize}

Успешными социальными сетями, как правило, пользутются несколько миллионов человек ежедневно и десятки миллионов пользователей в месяц. 
Количество зарегистрированных пользователей может достигать сотен миллионов.
Активные пользователи переодически создают данные ( сообщения, посты, комментарии... ).
Разумеется, что такой обьем данных физически не может уместиться на одном сервере.
К тому же пользователи, как правило, активно осуществляют доступ к контенту,
созданному другими пользователями.
Как правило этот контент представляется в виде ``лент'' - последовательностей событий, сагрегиронванных и отсортированных различным образом и изменяющихся во времени.

Целью данного курсового проекта является создание новостной социальной сети,
количество серверов для обеспечения работы которой 
можно было бы линейно увеличивать в зависимости от числа пользователей.
Так как социальная сеть состоит из достаточно большего числа компонентов,
реализация полностью масштабируемой системы является сложной задачей,
выходящей за рамки курсового проекта.
Требования масштабируемости предьявляются только для самих страниц лент и текста
публикаций в этих лентах.

Для достижения поставленной цели необходимо решить следующие задачи%
\begin{itemize}
\item провести аналитическое исспледования предметной области;
\item сконфигурировать все используемые хранилища данных;
%\item выявить нагрузочные требования к системе, способы доступа к сущьностям, количество обращений;
%\item подобрать для каждой сущьности хранилище, отвечающее требованиям и ``оптимальным образом'' организующее данные;
\item создать веб-приложение, осуществляющее доступ к данным.
\end{itemize}

%% Целью работы является создание всякой всячины. Для достижения поставленной цели необходимо решить следующие задачи:

%% \begin{itemize}
%% \item проанализировать существующую всячину;
%% \item спроектировать свою, новую всячину;
%% \item изготовить всякую всячину;
%% \item проверить её работоспособность.
%% \end{itemize}

%% Вот так-то. А этот абзац вставлен для визуальной оценки отступа от перечня до следующего абзаца.
