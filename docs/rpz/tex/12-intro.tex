\Introduction

В настоящее время существует большое разнообразие социальных сетей различных направлений. Основными потребностями пользователей данных продуктов являются:
\begin{itemize}
\item Общение по средством сообщений, комментариев...
\item Получение информации из определенных источников (публикаций других пользователей)
\end{itemize}

Успешными социальными сетями, как правило, пользутются несколько миллионов человек ежедневно и десятки миллионов пользователей в месяц. 
Количество зарегистрированных пользователей может достигать сотен миллионов.
Активные пользователи переодически создают данные ( сообщения, посты, комментарии... ).
Разумеется, что такой обьем данных физически не может уместиться на одном сервере.
К тому же пользователи, как правило, очень активно читают контент,
созданный другими пользователями.
Как правило этот контент представляется в виде ``лент'' - последовательностей событий, отсортированных различным образом и изменяющихся во времени.

Целью данного курсового проекта является создание новостной социальной сети,
количество серверов для обеспечения работы которой 
можно было бы линейно увеличивать в зависимости от числа пользователей.

%% Целью работы является создание всякой всячины. Для достижения поставленной цели необходимо решить следующие задачи:

%% \begin{itemize}
%% \item проанализировать существующую всячину;
%% \item спроектировать свою, новую всячину;
%% \item изготовить всякую всячину;
%% \item проверить её работоспособность.
%% \end{itemize}

%% Вот так-то. А этот абзац вставлен для визуальной оценки отступа от перечня до следующего абзаца.
